\documentclass[10pt,a4paper]{article}

\usepackage[utf8]{inputenc}
\usepackage[T1]{fontenc}

%%%%%%%%%%% Own packages
\usepackage[a4paper, margin=1in]{geometry}
\usepackage{multicol}
\usepackage{lipsum}

% Header/footer
\usepackage{fancyhdr}
\pagestyle{fancy}
\renewcommand{\headrulewidth}{0pt}


% Maths
%\usepackage{physics}
\usepackage{esdiff}
\usepackage{cancel}
\usepackage{amstext,amsbsy,amssymb,mathtools}
\usepackage{times} 
\usepackage{siunitx}
\usepackage{tensor}

%% Graphics
\usepackage{caption}
\captionsetup{margin=20pt,font=small,labelfont=bf}
%\renewcommand{\thesubfigure}{(\alph{subfigure})} % Style: 1(a), 1(b)
%\pagestyle{empty}
\usepackage{graphicx} % Include figure files

% Listsings and items
\usepackage[page]{appendix}
\usepackage[shortlabels]{enumitem}
\setenumerate{wide,labelwidth=!, labelindent=0pt}
\usepackage{varioref}
\usepackage{hyperref}
\usepackage{cleveref}

% Paragraph indent and skip
\setlength{\parindent}{2em}
\setlength{\parskip}{1em}
\setlength\extrarowheight{5pt}

%% User units
\DeclareSIUnit \parsec {pc}

%% User commands
\providecommand{\qwhere}
{
    \ensuremath{
    ,\quad \text{where} \quad 
    }
}

\title{AST5220 Cosmology \rm{II}\\ 
\vspace{5mm}Milestone 1 - The Background Cosmology}
\author{Jakob Borg}
%%%%%%%
\begin{document}
%%%%%%%

\maketitle

\lhead{Milestone 1 AST5220}
\rhead{Jakobbor}
%%%%%%%%

\section{Brief Introduction}
\label{sec:Intro}
In this first milestone\footnote{First of four milestones.} our goal is to solve for the evolution of the uniform background cosmology of our universe. With a set of cosmological parameters the numerical code developed here will use the Friedmann equation to produce the Hubble parameter and solve for the conformal time, both as functions of some time variable from the early Universe to today and beyond.

\section{Theory}
\label{sec:Theory}
\subsection{The Friedmann Equation}
\label{subsec:Theory/Friedmann}
Assuming a homogeneous and isotropic Universe, where each energy component act as a perfect fluid, we can derive the Friedmann equation from Einstein's theory of general relativity. Expressed in terms of density we write the equation as
%
\begin{equation}
    H^2 = \frac{8\pi G}{3} \sum_i \rho_i \qwhere H = \frac{1}{a}\diff{a}{t} = \frac{\dot{a}}{a}
    \label{eq:Friedmann_density}
\end{equation}
%
and the sum goes over each component contributing to the energy density of the Universe. In our case we consider the components baryons ($B$), cold dark matter ($CDM$), radiation ($R$), the cosmological constant ($\lambda$), curvature ($K$) and neutrinos ($\nu$). By using the critical density $\rho_c$, which is the density required for a perfectly flat universe, we introduce the relative density parameters $\Omega_i$ for each component
%
\begin{equation}
    \Omega_i = \frac{\rho_i}{\rho_c} \qwhere \rho_c = \frac{3H^2}{8\pi G}.
    \label{eq:Omega_i}
\end{equation}
%
In our close to perfectly flat universe we then have that the sum of the density parameters should equal unity
%
\begin{equation}
    \sum_i \Omega_i = 1.
    \label{eq:Omega_sum}
\end{equation}
%
To simplify the calculations we assume no contribution from the curvature, $\Omega_K=0$, and we neglect the tiny contribution from the neutrinos, $\Omega_\nu = 0$. Most of the today's value of these relative density parameters we can measure, and the last one may be obtained through the others using \cref{eq:Omega_sum}. In the end we want to express \cref{eq:Friedmann_density} in terms of these known parameters.

\subsection{Energy and Matter Densities}
\label{subsec:Theory/densities}
By conservation of mass and energy we also have the continuity equation for each component
%
\begin{equation}
    \dot{\rho_i} + 3H \left(\rho_i + P_i \right) = 0.
    \label{eq:continuity equation}
\end{equation}
%
Here we define the equation of state on the form $\omega = \frac{P}{\rho}$ and solve \cref{eq:continuity equation} for the density in terms of $\omega$ and the scale factor $a$. The solutions will be on the form
%
\begin{equation}
    \rho_i = \rho_{i,0} a^{-3(1+\omega_i)}
    \label{eq:rho_i(a,w)}
\end{equation}
%
where subscripts $0$ indicate today's values and the equation of state parameter for each component is known. The densities are presented in \cref{tab:densities} together with expressions for the relative density parameters.
%
\begin{table}[ht]
    \centering
    \begin{tabular}{|l|c|l|l|}
    \hline
    Component & $\omega$      & \multicolumn{1}{c|}{$\rho(a)$} & \multicolumn{1}{c|}{$\Omega(a)$}        \\[3pt] \hline
    $CDM$     & $0$           & $\rho_{CDM,0}a^{-3}$           & $\frac{H_0^2}{H^2}\Omega_{CDM,0}a^{-3}$ \\[3pt] \hline
    $B$       & $0$           & $\rho_{B,0}a^{-3}$             & $\frac{H_0^2}{H^2}\Omega_{B,0}a^{-3}$   \\[3pt] \hline
    $R$       & $\frac{1}{3}$ & $\rho_{R,0}a^{-4}$             & $\frac{H_0^2}{H^2}\Omega_{R,0}a^{-4}$   \\[3pt] \hline
    $\Lambda$ & $-1$          & $\rho_{\Lambda,0}$             & $\frac{H_0^2}{H^2}\Omega_{\Lambda,0}$   \\[3pt] \hline
    \end{tabular}
    \caption{Table of density and relative density expressions as functions of scale factor $a$ and present day values (subscript $0$) for each component. The expressions for the relative density parameters derived in \cref{asec:density_params}.}
    \label{tab:densities}
\end{table}
%
\subsection{Conformal Time}
\label{subsce:Theory/Conformal Time}
The last quantity we want, which will be the main focus of the computation in this milestone, is the conformal time. This is the maximum distance that light may have traveled since the big bang, which will be dependent on the expansion of the Universe. Here we used the differential form of the conformal time, defined as 
%
\begin{equation}
    \diff{\eta}{t} = \frac{c}{a}.
    \label{eq:eta_diff}
\end{equation}
%

\section{Method}
\label{sec:Method}
\subsection{Time Variable}
\label{subsec:Method/Time variable}
The default choice of the scale factor as time variable is used in the equations listed in \cref{sec:Theory}. In order to get the required time resolution in the rapidly evolving early universe all the way to today we instead use the logarithmic scale factor, $x= \ln(a)$, as our main time variable in all the calculations. The results will be presented in terms of this logarithmic time variable, as well as the redshift $z$ defined as
%
\begin{equation}
    1+z = \frac{1}{a} = \frac{1}{\exp(x)}.
    \label{eq:redshift}
\end{equation}

\subsection{Code Implementation}
\label{subsec:Method/Code Implementation}

\subsubsection{Cosmological Parameters}
\label{subsubsec:Method/Cosmological parameters}
The known cosmological parameters given as input to our code and are listed in \cref{tab:input_parameters}. These are used as a basis for all our calculations.
%
\begin{table}[ht]
    \centering
    \begin{tabular}{|l|l|}
    \hline
    Parameter      & Value    \\ \hline
    $h$            & $0.7$    \\ \hline
    $T_{CMB}$      & $2.7255$ \\ \hline
    $\Omega_{B,0}$     & $0.046$  \\ \hline
    $\Omega_{CDM,0}$ & $0.224$  \\ \hline
    \end{tabular}
    \caption{Cosmological parameters given as input to our code. Values taken from observations. Listed are the unitless Hubble constant $h$, the mean temperature of cosmic microwave background $T_{CMB}$ and the present day relative densities for baryons and cold dark matter.}
    \label{tab:input_parameters}
\end{table}
%

With this input we first find the Hubble parameter today and the remaining nonzero density parameters\footnote{The methods and calculations for the densities of neutrinos and curvature is still in the code to keep it general, for possible further expansion of the project to include these terms. As they stand they only return zero and will not affect the calculations.}
%
\begin{alignat}{2}
    H_0 &= 100h\,\si{\km\per\s\per\mega\parsec} &&= \SI{2.26855e-18}{s}\label{eq:H0}
    \\
    \Omega_{R,0} &= \frac{2\pi^2}{30}\frac{\left(k_bT_{CMB}\right)^4}{\hbar^3c^5}\frac{8\pi G}{3H_0^2} &&= \num{5.04688e-5} \label{eq:OmegaR0}
    \\
    \Omega_{\Lambda,0} &= 1 - \sum_{i\neq \Lambda} \Omega_i &&= \num{0.72995} \label{eq:OmegaLambda0}.
\end{alignat}
%
\subsection{Get Methods}
\label{subsec:Method/Get methods}
Using the preferred time variable, the expressions listed for the relative density parameters in \cref{tab:densities} are rewritten into functions of $x$ and implemented as separate methods.

The expressions for the densities in \cref{eq:Friedmann_density} are rewritten using the critical density, the present day values of the relative density parameters and their time dependence 
%
\begin{equation}
    \rho_i = \rho_{c,0} \Omega_{i,0} a^{-3(1+\omega_i)} = \frac{3H_0^2}{8\pi G}\Omega_{i,0} a^{-3(1+\omega_i)}.
\end{equation}
%
The final form of the Friedmann equation implemented as its own method in the code then reads
%
\begin{equation}
    H = H_0 \sqrt{\left(\Omega_{B,0}+\Omega_{CDM,0}\right)\exp(-3x) + \Omega_{R,0}\exp(-4x) + \Omega_{\Lambda,0}}.
    \label{eq:Friedmann implemented} 
\end{equation}

We also define the "Hubble prime" parameter
%
\begin{equation}
    \mathcal{H} = aH.
    \label{eq:Hprime}
\end{equation}
This is essentially the same as $\dot{a}$, and can be thought of as the rate of the expansion of the Universe. This quantity, together with its first and second derivative, will be used more in the milestones to come. For now we can use them to benchmark the implementation of our code, see \cref{asec:Benchmarking} for the expressions and the results of the testing. $\mathcal{H}$ is also used in the expression for the conformal time, which is the final part of our implementation.

\subsection{Solving the Conformal Time}
\label{subsec:Method/Conformal time}
By the chain rule we can rewrite \cref{eq:eta_diff} first into a differential equation in $a$, and then into $x$, as follows
\begin{alignat}{2}
    \diff{\eta}{t} &= \diff{\eta }{a}\diff{a}{t} = \diff{\eta }{a}\mathcal{H} = \frac{c}{a} & \qquad \Rightarrow \diff{\eta }{a} &= \frac{c}{a\mathcal{H}} \notag
    \\
    \diff{\eta }{x} &= \diff{\eta }{a}\diff{a}{x} \qwhere \diff{a}{x} = a & \qquad \Rightarrow \diff{\eta }{x} &= \frac{c}{\mathcal{H}(x).} \label{eq:eta_diff_of_x}
\end{alignat}
With this simple ordinary differential equation (ODE) we used the provided ODEsolver, wrapping GSL's Runge-Kutta 4 algorithm, to solve the system. As we are using the logarithmic scale factor we have some trouble defining our initial condition numerically, which should be $\eta(a=0)=0$. We cannot start our time variable $x$ at $a=0$, which corresponds to $x\to-\infty$, so instead we first used a wide range of x-values, $x\in[-25,3]$, to solve the system with an approximation for the initial condition $\eta(x=-25)=0$. Then we compute a spline of the result, and use this spline to evaluate the solution in a more well behaved range of values, $x\in[-15,2]$.

Finally all the results are evaluated in this smaller range of $x$-values using the implemented methods and plotted using "Python".

\section{Results}
\label{sec:Results}
First we present the evolution of the relative energy densities of the Universe, shown in \cref{fig:omegas_of_x}. Here each component is plotted, together with the combined contribution from all matter ($CDM + B$). We see three distinct regimes where each component is dominating, radiation domination marked with blue, matter domination marked with red and lastly the present day regime dominated by the cosmological constant marked in green. This color coding is used to better visualize these regimes in all the produced plots.

The main result from this milestone is shown in \cref{fig:hubbe_and_eta}. Here are the Hubble parameter against redshift and $x$ in the two upper panels. Note that the panel with redshift only goes to $z=10^{-2}$, chosen arbitrarily to indicate today. Here we see how the Hubble parameter decreases steeply in the radiation dominated era, before it enters the matter dominated era with a slightly gentler slope. This is also evident in the two lower panels. The conformal time is basically the inverse of the Hubble parameter, while the Hubble prime parameter has an even steeper decrease in the radiation dominated era going over to a less steep decrease in the matter dominated. We note that the Hubble prime starts to increase again going into the last regime, dominated by the dark energy component, which corresponds to an accelerated expansion like we observe today.

Note that the dashed lines in the lower left panel is the analytical solutions to \cref{eq:Friedmann implemented}, solved with only the dominated component in each regime and fitted to the data with a constant.

Solving for the conformal time took in total $\SI{0.00678444}{s}$, while the hole run of everything done in the milestone was done in $\SI{0.0586608}{s}$.

\begin{figure}[t]
    \centering
    \includegraphics[scale=0.5]{../figs/omegas_of_x.pdf}
    \caption{Plot showing relative density parameters against the logarithmic scale factor. The early Universe was dominated by radiation, marked in blue. Following is the matter dominated era is marked in red, where the combined density of the dark matter and baryon components is indicated by the dashed orange line. Lastly we have the present era dominated by the cosmological constant, marked in green.}
    \label{fig:omegas_of_x}
\end{figure}

\begin{figure}[ht]
    \centering
    \includegraphics[scale=0.5]{../figs/Hubble_eta_of_x.pdf}
    \caption{Plot showing the Hubble parameter against logarithmic scale factor and redshift, in two upper panels, and the conformal time and Hubble prime against $x$ in the two lower panels. Each dominating regime is indicated in blue, red and green color. The dashed lines in the lower left panel is the analytical approximation solutions obtained from solving the Friedmann equation with only one component, one solution for each dominating regime.}
    \label{fig:hubbe_and_eta}
\end{figure}

\clearpage
\begin{appendices}
\appendix
\section{Expressions for the relative density parameters}
\label{asec:density_params}
The expressions for the relative density parameters presented in \cref{tab:densities} is obtained in the following way.
\begin{align*}
    \Omega_i & = \frac{\rho_i}{\rho_c} = \frac{1}{\rho_c} \rho_{i,0} a^{-3(1+\omega_i)} \quad \bigg\rvert \cdot \rho_{c,0} = \frac{3H_0^2}{8\pi G}
    \\
    & = \frac{\rho_{c,0}}{\rho_c} \frac{\rho_{i,0}}{\rho_{c,0}} a^{-3(1+\omega_i)} \qwhere \frac{\rho_{i,0}}{\rho_{c,0}} = \Omega_{i,0} \quad \text{and} \quad \frac{\rho_{c,0}}{\rho_c} = \frac{H_0^2}{H^2}
    \\
    \Rightarrow \Omega_i (a) & = \frac{H_0^2}{H^2} \Omega_{i,0} a^{-3(1+\omega_i)} 
\end{align*}

\section{Benchmarking the Code}
\label{asec:Benchmarking}
Here we quickly include the figures used for benchmarking the code, without much further explanation. This is mostly implemented to be able to check that the code still works as intended in the milestones to come.

First we check that $\frac{\mathcal{H}}{\exp(x)H} =1$ for all $x$, shown in \cref{fig:bench_Hp_over_aH}.
\begin{figure}[ht]
    \centering
    \includegraphics[scale=0.5]{../figs/ratio_Hprime_aH.pdf}
    \caption{Ratio of Hubble prime over the product of the scale factor and the Hubble parameter, which should equal unity. Note that the oscillations around unity are of order $10^{-5}$.}
    \label{fig:bench_Hp_over_aH}
\end{figure}

Last we check the ratio of the second and first derivatives of the Hubble prime over the Hubble prime. The expressions used for the derivatives is derived using the chain rule and expressed in terms of the Hubble parameter and its derivatives, which again is expressed using the methods implemented for the Hubble parameter. This way we can check that the actual implemented methods are correct, even tho it might be possible to express them in a more direct and numerically more efficient way. The ratios end up as follows
\begin{align*}
    \frac{\mathcal{H}^\prime}{H} &= 1-\frac{N}{2}
    \\
    \frac{\mathcal{H}^{\prime\prime}}{H} &= \left(1-\frac{N}{2}\right)^2
\end{align*}
where $N$ is the prefactor for $x$ in the exponentials for each component in \cref{eq:Friedmann implemented}. That is $N={3,\,4\,0}$ for radiation, matter and Cosmological constant respectively. Result shown in \cref{fig:bench_Hp_and_derivatives}.
\renewcommand{\floatpagefraction}{.99}%
\renewcommand{\textfraction}{0}
\begin{figure}[t]
    \centering
    \includegraphics[scale=0.5]{../figs/ratio_Hprime_and_derivatives.pdf}
    \caption{Ratio of Hubble prime and its first and second derivatives.}
    \label{fig:bench_Hp_and_derivatives}
\end{figure}

Then we check that the ratio $\frac{\eta \mathcal{H}}{c}$ converges to unity back in time in the radiation dominated regime, shown in \cref{fig:bench_eta_rad_dom_converges}, where the solution of the conformal time is $\eta = \frac{c}{\mathcal{H}}$.

\begin{figure}[b]
    \centering
    \includegraphics[scale=0.5]{../figs/ratio_eta_Hp_over_c.pdf}
    \caption{Check that the ratio of the conformal time and Hubble prime over c converges to unity in the radiation dominated regime.}
    \label{fig:bench_eta_rad_dom_converges}
\end{figure}

\end{appendices}

\end{document}