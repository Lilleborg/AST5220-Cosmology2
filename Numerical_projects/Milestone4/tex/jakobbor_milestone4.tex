\documentclass[10pt,a4paper]{article}

\usepackage[utf8]{inputenc}
\usepackage[T1]{fontenc}

%%%%%%%%%%% Own packages
\usepackage[a4paper, margin=1in]{geometry}
\usepackage{multicol}
\usepackage{lipsum}
\usepackage{natbib}

% Header/footer
\usepackage{fancyhdr}
\pagestyle{fancy}
\renewcommand{\headrulewidth}{0pt}

% Footnotes at bottom of page
\usepackage[bottom,marginal]{footmisc}
% \setlength{\footnotemargin}{1.8em}

% Maths
\usepackage{physics}
\usepackage{esdiff}
\usepackage{cancel}
\usepackage{amstext,amsbsy,amssymb,mathtools}
\usepackage{times} 
\usepackage{siunitx}
\usepackage{tensor}

%% Graphics
\usepackage{caption}
\captionsetup{margin=20pt,font=small,labelfont=bf}
%\renewcommand{\thesubfigure}{(\alph{subfigure})} % Style: 1(a), 1(b)
%\pagestyle{empty}
\usepackage{graphicx} % Include figure files

% Listsings and items
\usepackage[page]{appendix}
\usepackage[shortlabels]{enumitem}
\setenumerate{wide,labelwidth=!, labelindent=0pt}
\usepackage{varioref}
\usepackage{hyperref}
\usepackage{cleveref}

% Paragraph indent and skip
\setlength{\parindent}{2em}
\setlength{\parskip}{1em}
\setlength\extrarowheight{5pt}

%% User units
\DeclareSIUnit \parsec {pc}

%% User commands
\providecommand{\qwhere}
{%
\ensuremath{
,\quad \text{where} \quad%
}
}
\providecommand{\rCDM}
{\ensuremath{
\textrm{CDM}
}
}

\providecommand{\gtilde}
{
  \ensuremath{
    \tilde{g}
  }
}

\providecommand{\hprime}
{
  \ensuremath{
    \mathcal{H}
  }
}

\allowdisplaybreaks
\title{AST5220 Cosmology \rm{II}\\ 
\vspace{5mm}Milestone 4 - Computing the CMB Power Spectrum}
\author{Jakob Borg}
%%%%%%%
\begin{document}
%%%%%%%
\maketitle
\lhead{Milestone 4 AST5220}
\rhead{Jakobbor}
%%%%%%%%

\section{Introduction}
\label{sec:Introduction}
This is the final milestone of our four part project to compute the cosmic microwave background (CMB) power spectrum, expressed through $C_\ell$. Here we build upon the results from the three previous milestone where we computed the background evolution and expansion of the Universe \citep{milestone1}, the ionization history, the optical depth and visibility function \citep{milestone2} and last the evolution of the perturbations set up by inflation \citep{milestone3}.


The full project can be found on GitHub, here are link to the \href{https://github.com/Lilleborg/AST5220-Cosmology2/tree/master/Numerical_projects}{projects main directory}\footnote{URL: \url{https://github.com/Lilleborg/AST5220-Cosmology2/tree/master/Numerical_projects}}. All computation and calculations are done in "C++", with the source code found \href{https://github.com/Lilleborg/AST5220-Cosmology2/tree/master/Numerical_projects/src}{here}\footnote{URL: \url{https://github.com/Lilleborg/AST5220-Cosmology2/tree/master/Numerical_projects/src}}. 
The main calculations for this milestone may be found in \textit{PowerSpectrum.h}, building upon everything we have done so far. As mentioned in \cite{milestone3} we will need one specific quantity, the source function, which we will implement in the code from the last milestone in \textit{Perturbations.h}.


%  _______ _                           
% |__   __| |                          
%    | |  | |__   ___  ___  _ __ _   _ 
%    | |  | '_ \ / _ \/ _ \| '__| | | |
%    | |  | | | |  __/ (_) | |  | |_| |
%    |_|  |_| |_|\___|\___/|_|   \__, |
%                                __/ |
%                               |___/ 
\section{Theoretical Background}
\label{sec:Theory}

\subsection{The CMB Power Spectrum}
\label{subsec:Theory/CMB power spectrum}
The CMB power spectrum is the main goal of our calculations, described through $C_\ell$. This gives us a statistical representation of how much the different perturbations of varying scales contributes to the temperature map we observe when measuring the CMB. The angular scales $\ell$ are today related to the physical scale in Fourier space characterized through the wavenumber of the perturbations $k$, following
\begin{equation*}
  \ell \sim k \eta_0
\end{equation*}
where $\eta_0$ is the conformal time today.

The measured temperature can be expanded into spherical harmonics as
\begin{equation}
  T(\hat{n}) = \sum a_{\ell m} Y_{\ell m}(\hat{n})
  \label{eq: T spherical harmonics}
\end{equation}
where $Y_{\ell m}(\hat{n})$ are the spherical harmonics and $a_{\ell m}$ are the coefficients. The power spectrum it self is defined as the expectation value of the squared coefficients, which is the same as the variance as the coefficients are gaussian with zero expectation value,
\begin{equation*}
  C_\ell = \left<\abs{a_{\ell m}}^2\right> = \left<a_{\ell m}a_{\ell m}^\ast\right>.
\end{equation*}
But instead of finding the $a_{\ell m}$s we can use what we have from \cite{milestone3}, where we expressed the temperature of the CMB $T(\hat{n})$ in terms of the photon multipoles $\Theta_\ell$. Expressing the square of $a_{\ell m}$s through the square of the multipoles, using the primordial power spectrum to scale the multipoles and integrating over all the spatial directions we find the full expression for the power spectrum
\begin{equation}
  C_\ell = 4\pi \int_0^\infty A_s \left(\frac{k}{k_{\rm{pivot}}}\right)^{n_s-1}\frac{\Theta_\ell^2(k,x=0)}{k}\dd{k}
  \label{eq: C_ell} 
\end{equation}
where we have assumed an isotropic Universe, the scaling is done to adjust the simplified initial conditions for the differential equation system for the multipoles as discussed in \cite{milestone3} and
\begin{equation}
  \frac{k^3}{2\pi^2}P_{\rm{primordial}}(k) = A_s\left(\frac{k}{k_{\rm{pivot}}}\right)^{n_s-1}.
  \label{eq: Priomrodial power spectrum equation}
\end{equation}
Here $A_s$ is the amplitude, $n_s$ is the spectral index and $k_{\rm{pivot}}$ is some scale for which the amplitude of the spectrum is $A_s$. As we want to reproduce the power spectrum as we see it today, we use the todays value of the multipoles at $x=0$.

\subsubsection{The Line of Sight Integration}
\label{subsubsec:Theory/LOS int}
In order to solve \cref{eq: C_ell} from large scales, low $\ell$, to small scales, high $\ell$, we need all the multipoles for the corresponding scales we whish to compute. As discussed in \cite{milestone3} this is not solved through the endless hierarchy of Boltzmann equations for the multipoles. Instead we utilize the line of sight integration approach of Zaldarriaga and Seljak, which formally integrates the equation for $\dot{\Theta}$ and does the multipole expansion at the end. After some details this gives us a final expression for the different multipoles, which can be solved directly without the hierarchy
\begin{equation}
  \Theta_\ell (k, x=0) = \int_{-\infty}^0 \tilde{S}(k,x)j_\ell\left(k\eta_0 - k\eta\right)\dd{x}.
  \label{eq: LOS integration}
\end{equation}
Here $j_\ell$ is the spherical Bessel function evaluated at $k\eta_0 - k\eta(x)$, which projects the 3D field of the perturbations onto the 2D sphere we observe. $\tilde{S}$ is the aforementioned source function, which describes the physical effects changing the photons energies as it travels through the Universe to us. The source function is defined as 
\begin{equation}
  \tilde{S}(k,x) = \tilde{g}\left[ \Theta_0 + \Psi + \frac{1}{4}\Pi\right] +
  e^{-\tau} \left[\Psi^\prime-\Phi^\prime\right] -
  \frac{1}{ck}\frac{d}{dx}(\mathcal{H}\tilde{g}v_b) + \frac{3}{4c^2k^2} \frac{d}{dx}
  \left[\mathcal{H}\frac{d}{dx} (\mathcal{H}\tilde{g}\Pi)\right]
\end{equation}
following \cite{Calin}. All the involved quantities for the source function is calculated and discussed in the previous milestones. As discussed in \cite{milestone3} we don't include polarization in our computations, so $\Pi \equiv \Theta_2$. The last term can be rewritten using the chain rule to
\begin{align*}
  \frac{d}{dx}
  \left[\mathcal{H}\frac{d}{dx} (\mathcal{H}\tilde{g}\Pi)\right] &= \Pi \gtilde \left(\hprime \ddot{\hprime} - \dot{\hprime}^2\right) \\
  &\quad + 3\hprime\dot{\hprime}\left(\dot{\gtilde}\Pi+\gtilde\dot{\Pi}\right)\\
  &\quad + \hprime^2\left(\ddot{\gtilde}\Pi + 2\dot{\gtilde}\dot{\Pi} + \gtilde\ddot{\Pi}\right).
\end{align*}
The physical effects described by the source function can be seen from the four different terms
\begin{enumerate}[label=\arabic*]
  \item The Sachs Wolfe effect, the main contribution to the power spectrum, is the effective monopole $\Theta_0^{\rm{eff}} = \Theta_0 + \Psi + \frac{1}{4}\Pi$, describing the average temperature of the photons with the effect of climbing out of the gravitational potential $\Psi$ and with a small correction due to polarization in $\Pi$. This term is weighted by the visibility function, which essentially works as a Dirac Delta function at the time of recombination, effectively picking out the value of the effective monopole at the last scattering surface (LSS). This makes sense as we know the photons decouple after recombination and free steam to us, so we are actually observing the effective monopole at LSS.
  \item The integrated Sachs wolf effect, the effect on photons traveling through gravitational potentials that are changing in time. As we saw in \cite{milestone3} this is non-negligible during the period where small scale perturbations enter the horizon right before recombination, and in later times when large scale perturbations enter the horizon in the dark energy dominated era. As the term is weighted with the exponential of the optical depth, $e^{-\tau}$, this term is not contributing much before recombination, where $\tau > 1$, and thus the late integrated Sachs Wolf effect, from the largest scales, is the main contributor from this term.
  \item The third term is a doppler term, which describes the doppler effect on the photons from the slightly different peculiar velocities the perturbations have in the tight coupling regime, where photons and baryons are coupled and in thermal equilibrium due to the high optical depth. This term is again weighted by the visibility function and its derivatives, effectively giving us a contribution from this term at the LSS.
  \item The last term is small, and describes a small quadrupolar correction to the source function, again weighted by the visibility function and its derivatives.
\end{enumerate}

\subsection{Matter Power Spectrum}
\label{subsubsec:Theory/Matter power spectrum}
Last we have the matter power spectrum defined as
\begin{equation}
  P_i(k,x) = \abs{\Delta_i(k,x)}^2P_{\rm{primordial}}(k)
  \label{eq: Component matter spectrum}
\end{equation}
where $P_{\rm{primordial}}(k)$ is the primordial power spectrum from \cref{eq: Priomrodial power spectrum equation} and $\Delta_i$ is the gauge invariant density perturbation defined as
\begin{equation}
  \Delta_i = \delta_i - \frac{3(1+\omega_i)\hprime}{ck}v_i
  \label{eq: invariant delta}
\end{equation}
where $\omega_i$ is the equation of state for component $i$. Our main goal is the full matter power spectrum, for components $i=$ baryons and \rCDM. For the total matter we have the invariant density
\begin{equation}
  \Delta_M \equiv \frac{c^2k^2\Phi(k,x)}{\frac{3}{2}\Omega_{M,0}a^{-1}H_0^2}.
  \label{eq: invariant matter delta}
\end{equation}
For these results we also want to point out the equality scale, $k_{\rm{eq}}$. This marks the scale corresponding to the peak in the power spectrum due to the smaller scales entering the horizon in the radiation dominated regime being suppressed by the Meszaros suppression. The equality scale is found by
\begin{equation}
  k_{\rm{eq}} = \frac{a_{\rm{eq}}H(a_{\rm{eq}})}{c},
  \label{eq: k_eq}
\end{equation}
where $a_{\rm{eq}}$ is defined as the time where $\Omega_{R} = \Omega_{M}$, and can be approximated as $a_{\rm{eq}} \approx \frac{\Omega_{R,0}}{\Omega_{M,0}}$. The last approximation can be found by assuming only radiation and matter as energy contributors in the Universe, which is reasonable in the early Universe where the equality happens.

%  __  __      _   _               _ 
% |  \/  |    | | | |             | |
% | \  / | ___| |_| |__   ___   __| |
% | |\/| |/ _ \ __| '_ \ / _ \ / _` |
% | |  | |  __/ |_| | | | (_) | (_| |
% |_|  |_|\___|\__|_| |_|\___/ \__,_|
\section{Method}
\label{sec:Method}

\subsection{Data Grids and Resolution}
\label{subsec:Method/grid and resolution}

\subsection{Precalculating The Spherical Bessel Function}
\label{subsubsec:Method/Bessel function}

\subsection{Solving the CMB Power Spectrum}
\label{subsec:Method/Solve CMB}

\subsubsection{Line of Sight Integration}
\label{subsubsec:Method/LOS integration}

\subsubsection{$C_\ell$}
\label{subsubsec:Method/C_ell}
$A_s = 2e-9$
$n_s = 0.96$
$kpivot_mpc = 0.05$

\subsection{Solving the Matter Power Spectrum}
\label{subsec:Method/solving matter power spectrum}

\subsubsection{Equality Scale}
\label{subsubsec:Method/Equality Scale}

%  _____                 _ _       
% |  __ \               | | |      
% | |__) |___  ___ _   _| | |_ ___ 
% |  _  // _ \/ __| | | | | __/ __|
% | | \ \  __/\__ \ |_| | | |_\__ \
% |_|  \_\___||___/\__,_|_|\__|___/
\section{Results}
\label{sec:Results}
?? TIMING OF CALCULATIONS ??

?? Test of source and bessel functions ?? Compare to integrand figure in callin ??

\subsection{Transfer Function and Integrand}
\label{subsec:Results/Transfer and integrand}

\subsection{The reproduced CMB Power Spectrum}
\label{subsec:Results/CMB power spectrum}

\subsubsection{The constituent CMB terms}
\label{subsubsec:Results/CMB power spectrum terms}

\subsection{The Matter Power Spectrum}
\label{subsec:Results/Matter power spectrum}

\section{Conclusion}
\label{sec:Conclusion}
%\pagebreak
\bibliographystyle{plainnat}
\bibliography{ref_milestone4}

\end{document}